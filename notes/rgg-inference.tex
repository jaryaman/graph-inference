% Preamble ==================================================================
\documentclass[11pt]{article}
\usepackage{geometry}
\geometry{verbose,tmargin=2.5cm,bottom= 1.5cm,lmargin=2.5cm,rmargin=2.5cm}
\usepackage{float}
\usepackage{graphicx}
\usepackage{amsmath}
\usepackage{amssymb}
\usepackage{enumitem}
\usepackage{mathtools}
\usepackage{mathrsfs}

\usepackage{tensor}
\usepackage{cancel}
\usepackage{wasysym}
\usepackage{braket}

\usepackage{amsthm} % theorem

\numberwithin{equation}{section}

\usepackage{titlesec,dsfont}

%Format section heading style
\usepackage{sectsty}
\sectionfont{\sffamily\bfseries\large}
\subsectionfont{\sffamily\normalsize\slshape}
\subsubsectionfont{\sffamily\small\itshape}
\paragraphfont{\sffamily\small\textbf}


%Put period after section number
\makeatletter
\def\@seccntformat#1{\csname the#1\endcsname.\quad}
\makeatother

%Bibliography
\usepackage[round]{natbib}
\bibliographystyle{genetics}

%Format captions
\usepackage[ labelsep=period, justification=raggedright, margin=10pt,font={small},labelfont={small,normal,bf,sf}]{caption}

\setlength{\parskip}{0ex} %No space between paragraphs.

\renewcommand{\familydefault}{\sfdefault}

\newcommand\indep{\protect\mathpalette{\protect\independenT}{\perp}}
\newcommand{\nindep}{\not\!\perp\!\!\!\perp}
\def\independenT#1#2{\mathrel{\rlap{$#1#2$}\mkern2mu{#1#2}}}

%PUT ME LAST--------------------------------------------------
\usepackage[colorlinks=true
,urlcolor=blue
,anchorcolor=blue
,citecolor=blue
,filecolor=blue
,linkcolor=black
,menucolor=blue
,linktocpage=true
,pdfproducer=medialab
,pdfa=true
]{hyperref}

\makeatother %Put this last of all

% Symbol definitions
\newcommand{\defeq}{\coloneqq}
\renewcommand{\d}[1]{\ensuremath{\operatorname{d}\!{#1}}}
\newcommand{\dpow}[2]{\ensuremath{\operatorname{d}^{#2}\!{#1}}}
\newcommand{\deriv}[2]{\frac{\ensuremath{\operatorname{d}\!{#1}}}{\ensuremath{\operatorname{d}\!{#2}}}}
\newcommand{\derivn}[3]{\frac{\ensuremath{\operatorname{d}^{#1}\!{#2}}}{\ensuremath{\operatorname{d}\!{#3}^{#1}}}}
\DeclareMathOperator{\diag}{diag}
\DeclareMathOperator{\tr}{tr}
\newcommand{\tn}[2]{\tensor{#1}{#2}}

% Make theorems bold
\makeatletter
\def\th@plain{%
  \thm@notefont{}% same as heading font
  \itshape % body font
}
\def\th@definition{%
  \thm@notefont{}% same as heading font
  \normalfont % body font
}
\makeatother

% Theorem definitions
\newtheorem{thm}{Theorem}[section]
\newtheorem{defn}{Definition}[section]
\newtheorem{cor}{Corollary}[section]
\newtheorem{prop}{Property}[section]
\newtheorem{rle}{Rule}[section]
\newtheorem{lma}{Lemma}[section]

\newcommand{\bs}{\boldsymbol}

%Preamble end--------------------------------------------------


\begin{document}



\begin{flushleft}
\textbf{\Large RGG inference}
\end{flushleft}

\begin{flushleft}
Author: Juvid Aryaman

Last compiled: \today
\end{flushleft}

\section{Models}

\subsection{Soft random geometric graph}

Define the spatial Bernoulli graph as:
\begin{equation}
P(Y=y|D) = \prod_{i,j} B(Y_{ij} = y_{ij}| \mathcal{F}(D_{ij}, \theta))
\end{equation}
where $Y$ is the random graph adjacency matrix, $D$ is a matrix of inter-vertex distances, $B$ is the Bernoulli pmf, and $\mathcal{F}$ is a function taking distances into the $[0,1]$ interval. The log likelihood is
\begin{equation}
l = \sum_{ij}[ y_{ij}\log(\mathcal{F}(D_{ij})) + (1-y_{ij})  \log(1 - \mathcal{F}(D_{ij})   ]
\end{equation}
\noindent Some choices of $\mathcal{F}(D_{ij}, \theta)$ include:
\begin{enumerate}
\item Exponential: $\mathcal{F}(D_{ij}, \theta) = e^{-\lambda_r D_{ij}}$
\item Sigmoidal: $\mathcal{F}(D_{ij}, \theta) = 1/(1 + e^{\lambda_r D_{ij}})$
\end{enumerate}
We can do inference in Pymc3 easily in either case.

\subsection{Poissonian geometric graph}
We define a Poissonian geometric graph as a multigraph with no self-edges, where the expected number of edges between nodes $i$ and $j$ is $\mathcal{F}(D_{ij}, \theta)$. The likelihood may be written as:
\begin{equation}
P(G|D, \theta) = \prod_{i<j} \frac{(\mathcal{F}(D_{ij}, \theta))^{A_{ij}}}{A_{ij}!} \exp(-\mathcal{F}(D_{ij}, \theta))
\end{equation}
\noindent Some choices of $\mathcal{F}(D_{ij}, \theta)$ include:
\begin{enumerate}
\item Exponential: $\mathcal{F}(D_{ij}, \theta) = \lambda_0 e^{-\lambda_r D_{ij}}$
\item Sigmoidal: $\mathcal{F}(D_{ij}, \theta) = \lambda_0/(1 + e^{\lambda_r D_{ij}})$
\end{enumerate}

\subsection{Degree-corrected Poissonian geometric graph}
We define a degree-corrected Poissonian geometric graph as a multigraph with no self-edges, where the expected number of edges between nodes $i$ and $j$ is $k_i k_j \mathcal{F}(D_{ij}, \theta)$. The likelihood may be written as:
\begin{equation}
P(G|D, \theta) = \prod_{i<j} \frac{(k_i k_j \mathcal{F}(D_{ij}, \theta))^{A_{ij}}}{A_{ij}!} \exp(- k_i k_j \mathcal{F}(D_{ij}, \theta))
\end{equation}
\noindent Some choices of $\mathcal{F}(D_{ij}, \theta)$ include:
\begin{enumerate}
\item Exponential: $\mathcal{F}(D_{ij}, \theta) = \lambda_0 e^{-\lambda_r D_{ij}}$
\item Sigmoidal: $\mathcal{F}(D_{ij}, \theta) = \lambda_0/(1 + e^{\lambda_r D_{ij}})$
\end{enumerate}
We interpret $k_i$ as ``KOL-ness''.

\newpage



\section*{Literature}

\section{Garrod (2020)}
In \citep{Garrod20}, the probability of individuals $i$ and $j$ sharing a social tie is given by
\begin{equation}
P(A_{ij} = A_{ji} = 1| x_i, x_j, \theta) = f(d(x_i, x_j), \theta)
\end{equation}
where $d(x_i, x_j)$ is the social distance metric between individuals $i$ and $j$, and $\theta$ are parameters. This can be Euclidean distance if we only have geographic position. For geographic distance, it is common to consider exponential or power law decay \citep{Daraganova12}. A typical choice might be
\begin{equation}
f(d(x_i, x_j), \theta) = \frac{1}{1 + e^{d(x_i, x_j)}}.
\end{equation}
Often, one reduces soft RGGs down to a model in which each node belongs to one of a finite number of blocks, with specified probabilities connecting within and between these blocks, called a stochastic block model.

Tuning the embedding space, distance metric, and connection probability allow you to create networks with different properties. For instance, embedding in a hyperbolic space allows us to generate graphs with heterogeneous degree distributions. Embedding graphs in Lorentzian spacetime (i.e. one dimension being time) allows us to generate directed networks; this is useful for e.g. citation networks.

\section{Butts (2012)}
In \citep{Butts12}, a spatial Bernoulli graph is defined as 
\begin{equation}
P(Y=y|D) = \prod_{i,j} B(Y_{ij} = y_{ij}| \mathcal{F}(D_{ij}, \theta))
\end{equation}
\noindent where $Y$ is the random graph adjacency matrix, $D$ is a matrix of inter-vertex distances, $B$ is the Bernoulli pmf, and $F$ is a function taking distances into the $[0,1]$ interval.


\section{Karrer (2011)}
\citep{Karrer11} defines a degree-corrected stochastic blockmodel. 

\subsection{Standard (Poissonian) stochastic blockmodel}
For a ``standard'' stochastic blockmodel, they allow their network to contain \textbf{multiedges}: i.e. more than one edge between any two nodes: this is for computational convenience. Denote
\begin{enumerate}[nosep]
\item $\text{Poisson}(\psi_{rs})=$ number of edges between nodes $r$ and $s$. This permits self-edges.
\item $A_{ij} = $ number of edges between $i$ and $j$; $A_{ii} = $ \textbf{twice} the number of self-edges for node $i$
\item $\omega_{rs}= \mathbb{E}(A_{ij})$ given that node $i$ exists in group $r$ and node $j$ exists in node $s$
\item $g_i=$ group assignment of node $i$
\end{enumerate}
\noindent The likelihood is therefore
\begin{equation}
P(G|\omega, g) = \prod_{i<j} \frac{(\omega_{g_i g_j})^{A_{ij}}}{A_{ij}!}\exp(- \omega_{g_i g_j}) \times \prod_{i} \frac{(\frac{1}{2}\omega_{g_i g_i})^{A_{ii}/2}}{(A_{ii}/2)!}\exp(- \frac{1}{2} \omega_{g_i g_i}).
\end{equation}

Define 
\begin{equation}
m_{rs} = \sum_{ij} A_{ij} \delta_{g_i, r} \delta_{g_j, s}
\end{equation}

\subsection{Degree-corrected stochastic blockmodel}
Introduce a new set of parameters $\theta_i$ controlling the expected degrees of vertices $i$. Let
\begin{equation}
\mathbb{E}(A_{ij}) = \theta_i \theta_j \omega_{g_i g_j}.
\end{equation}
Hence
\begin{equation}
P(G|\omega, g) = \prod_{i<j} \frac{(\theta_i \theta_j \omega_{g_i g_j})^{A_{ij}}}{A_{ij}!}\exp(- \theta_i \theta_j \omega_{g_i g_j}) \times \prod_{i} \frac{(\frac{1}{2} \theta_i^2 \omega_{g_i g_i})^{A_{ii}/2}}{(A_{ii}/2)!}\exp(- \frac{1}{2} \theta_i^2 \omega_{g_i g_i})
\end{equation}
The $\theta$ parameters are arbitrary to within a multiplicative constant that is absorbed into the $\omega$ parameters. The normalization of $\theta$ can be fixed by imposing the constraint
\begin{equation}
\sum_i \theta_i \delta_{g_i r} = 1
\end{equation}
for all groups $r$. Hence $\theta_i$ is the probability that an edge connected to the community which $i$ belongs lands on vertex $i$ itself [\textbf{TODO}: Don't understand]. With this constraint, the log likelihood simplifies to (up to constant factors)
\begin{equation}
\log P(G|\theta, \omega, g) = 2 \sum_i k_i \log \theta_i + \sum_{rs} (m_{rs} \log \omega_{rs} - \omega_{rs})
\end{equation}

\subsection{To read}
\begin{enumerate}
\item \url{https://git.skewed.de/count0/graph-tool}
\item This paper is very similar to the Poissonian RGG \citep{Peixoto20}
\end{enumerate}



\newpage
\bibliography{rgg-inference.bib} 

\end{document}
